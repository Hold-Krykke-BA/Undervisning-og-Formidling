\lipsum[5] \cite{einstein}. 

\subsection{Billede og caption}
\label{sec:1.1}
\lipsum[10]

\begin{figure}[H]
    \centering
    \caption*{Flot illustration med tekst foroven}
    \includegraphics[width = 0.8\textwidth ]{figures/this_is_fine.png}
    \caption{Og tekst forneden. Ingen grund til panik. \cite{menabrea}}
    \label{fig:fine}
\end{figure}

\subsection{Flere billeder og henvisninger!}
\label{sec:1.2}

\begin{figure}[H]
\centering
\begin{subfigure}{.5\textwidth}
  \centering
  \includegraphics[width=.4\linewidth]{figures/pepe_study.png}
  \caption{Study hard}
  \label{fig:study}
\end{subfigure}%
\begin{subfigure}{.5\textwidth}
  \centering
  \includegraphics[width=.4\linewidth]{figures/pythonk.png}
  \caption{Get very lost}
  \label{fig:thonk}
\end{subfigure}
\caption{En figur med to billeder!}
\label{fig:two_img}
\end{figure}

Se i figur~\ref{fig:two_img} på side~\pageref{fig:two_img} (lige oven for) for en grundig illustration af hvordan arbejdet med at forstå \LaTeX\ er. 

\subsubsection{Subsubsections, paragraphs, subparagraphs}
\label{sec:1.2.1}
\paragraph{Paragraf}
\lipsum[2]
\subparagraph{Subparagraf}
\lipsum[2]






