
Jeg fandt det relativt svært at finde meget specifikke krav, såsom antal af studerende per bacheloropgave, krav til virksomheden med mere. Det jeg har kunne finde fra formelle kilder er mere generelt og overordnet og beskriver uddannelsen som helhed. 

\vspace{0.5cm}
\subsection{Formelle kilder}
\label{sec:2.1}

I den gældende studieordningen af 2017 for Copenhagen Business Academy står der:
\begin{displayquote}
\emph{Bachelorprojektet skal dokumentere den studerendes forståelse af og evne til at
reflektere over professionens praksis og anvendelse af teori og metode i relation til en
praksisnær problemstilling. Problemstillingen, der skal være central for uddannelsen og
professionen, formuleres af den studerende, eventuelt i samarbejde med en privat eller
offentlig virksomhed. Cphbusiness godkender problemstillingen \cite{cph}.} 
\end{displayquote}
\vspace{0.5cm}
Herudover oplistes følgende relevante punkter for bacheloropgaven:
\begin{itemize}
    \item Omfanget er 15 ECTS point
    \item 3. semester
    \item Mundtlig og skriftlig prøve - samlet karakter
    \item Ekstern censur
    \item Prøven kan først finde sted efter uddannelsens øvige prøver og praktik er bestået
\end{itemize}

Læringsmål beskrives som: 
\begin{displayquote}
\emph{Det afsluttende bachelorprojekt skal dokumentere, at uddannelsens afgangsniveau er opnået, jf. kapitel 1 i dette dokument \cite{cph}.} 
\end{displayquote}
\vspace{0.5cm}
I Studieordningen er der en henvisning til BEK nr. 247 af 15/03/2017: Bekendtgørelse om tekniske og merkantile erhvervsakademiuddannelser og professionsbacheloruddannelser \cite{bek}, samt Erhvervsakademiernes uddannelsesnetværk \cite{nek}.   

Begge henvisninger underbygger studieordningen og kommer ikke med yderligere opklaring i forhold til specifikke krav til bachelorprojektet. 


\newpage
\vspace{0.5cm}
\subsection{Kick-off information}
\label{sec:2.2}

Der afholdes kick-off møde inden 3. semester og på dette blev der i November 2020 delt en powerpoint \cite{sli} og et dokument \cite{pdf} der konkretiserer de krav der er til selve bacheloropgaven. 

\subsubsection{Konkrete krav til bacheloropgaven}
\label{sec:2.2.1}

\begin{itemize}
    \item Den studerende skal udvikle et projekt hvori der:
        \begin{itemize}
            \item undersøges en softwarerelateret problemstilling.
            \item foreslåes en løsning.
            \item implementeres en løsning.
        \end{itemize}
    \item Den studerende skal såvidt det er muligt inkludere elementer fra uddannelsens fag.
    \item Bacheloropgaven må skrives i grupper på op til 4 studerende.
    \item Det maksimale sideantal er givet ved: 40 + 20 · antalStuderende.
    \item Der er intet minimum sideantal, dog beskrives opgaver koretere end to tredjedele af maksimum som korte. 
    \item Der må skrives på dansk eller engelsk. 
\end{itemize}
\vspace{0.5cm}

Herudover er der en del anbafalinger til indhold i bacheloropgaven, men da dette ikke er specifikke krav tages de ikke med i denne aflevering. 