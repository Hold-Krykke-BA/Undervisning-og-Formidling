\documentclass{article}
\usepackage{url} %bibtex
\usepackage{hyperref} %bibtex
\setcounter{secnumdepth}{0} % remove section numbers from ToC, sections are level 1
\usepackage{csquotes} %quotation styles
\usepackage{cite}


\title{Requirements of bachelor thesis}
\date{} % remove date from \maketitle
\author{Rúni VN}

\begin{document}
\maketitle

\begin{center}
\emph{Find requirements of bachelor thesis.\\ 
Write a \LaTeX\  document explaining your findings.\\
Document your sources.}

\tableofcontents

\section{Discussion}
\textbf{The only supplied information} was that some answers could be found in the curriculum.

\section{Procedure}
\textbf{Looking at the curriculum}, not much was to be found, as it and most of its references were rewritings of legal documents from \url{retsinformation.dk}.
Instead, I looked to its useful references, Google, my study group and descriptions of the bachelor assignment from last year.\\
\vspace{5mm}
It was easy to find guides on \emph{how to write a bachelor thesis} - but that was not the goal here, it was the requirements for one.\\
Finding answers was harder than I thought, especially because I wanted to find answers related to the bachelor thesis of Software Development (rather than another study).

\section{Findings}

\blockquote[\cite{cphCurriculum,deaCurriculum}][\emph{The bachelor’s project must document the student’s understanding of and ability to
reflect on the practices of the profession and the use of theory and methods in
relation to a real-life problem. The problem statement, which must be central to the
programme and profession, is formulated by the student, possibly in collaboration
with a private or public company. The academy approves the problem statement.}]

\textbf{The only formal requirements} I managed to find are the following:

\begin{itemize}
    \item [$\textendash$] The project may be done in groups of up to 4 people.
    \item [$\textendash$] The maximum page count for the bachelor report is given by the formular \(maxPageCount = 40 + (20 \cdot numberOfStudents)\)
    \item [$\textendash$] The report can be written in either Danish or English
    \item [$\textendash$] The report should contain a thorough description of the work that
    has been done during the bachelor project, as well as an evaluation and reflection
    on the work.
    \cite{pdf_kickoff_2020}
\end{itemize}

\textbf{Additionally}, there are a number of legal documents and writings related to the bachelor programme itself, of which the following is noteworthy:
\begin{itemize}
    \item [$\textendash$] The bachelor thesis project is set to 15 ECTS. Earlier revisions show the option between 10, 15 or 20 ECTS. \cite{retsinf_1162}
    \item [$\textendash$] Formulation and spelling is part of the evaluation too:
\end{itemize}

\blockquote[\cite{retsinf_18}][\emph{\S 37. Stk. 4. Ved bedømmelsen af professionsbachelorprojekt, afsluttende eksamensprojekt eller afgangsprojekt skal der ud over det faglige indhold også lægges vægt på den studerendes formulerings- og staveevne.}]

\textbf{On top of that}, the curriculum states that there is an actual manual available:\\

\blockquote[\cite{cphCurriculum}][\emph{(...)  For more see the Manual for the Bachelor Project for the study programme.}]

\underline{I am, however, unable to find it.}

\section{Conclusion}
\textbf{It seems that there aren't many hard requirements}, neither from the government nor the school.\\
The government allots either 10, 15 or 20 ECTS point to the bachelor project and the school generally supplies guidelines over rules.\\
Honorable mention, which didn't make it into this paper: The kick-off slides for last semesters bachelor project: \cite{kickoffSlides}

\end{center}

%\renewcommand\refname{\vskip -1cm} % Remove "references" from bibliography
\bibliographystyle{apalike}
\bibliography{ba_requirements} 

\end{document}