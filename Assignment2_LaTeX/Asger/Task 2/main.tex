\documentclass{article}
\usepackage[utf8]{inputenc}
\usepackage{graphicx}
\graphicspath{ {./images/} }
\usepackage[export]{adjustbox}
\usepackage{subcaption}
\usepackage{float}
\usepackage{placeins}
\usepackage{amssymb}
\usepackage{listings}
\usepackage{color}
\usepackage{amsmath}
\usepackage{url}
\usepackage[english]{babel}
\usepackage[backend=biber]{biblatex}
\addbibresource{bibliography.bib}
\usepackage[nottoc]{tocbibind}
\usepackage{todonotes} % Change to \usepackage[disable]{todonotes} to hide all notes without deleting them

\title{Latex Assignment \\
 \large Exploration and Presentation}
\author{Asger Thorsboe Lundblad}
\date{February 2021}

\begin{document}

\maketitle

\tableofcontents
\listoffigures

\section{Testing æ å ø}
Dealing with Æ, æ, Å, å, Ø and ø.

\section{Dealing with images}
\FloatBarrier
\includegraphics[width=\textwidth, height=10cm]{Alphabet41393.jpg}

\begin{figure}[h]
    \centering
    \caption{Caption over image}
    \includegraphics[width=12cm, height=10cm]{Alphabet41393.jpg}
    \caption{Caption under image}
    \label{fig:fig1}
\end{figure}

Praesent blandit blandit mauris. Praesent
lectus tellus, aliquet aliquam, luctus a, egestas a, turpis. Mauris 
lacinia loremsit amet ipsum. Nunc quis urna dictum turpis accumsan 
semper.In figure~\ref{fig:fig2} on page~\pageref{fig:fig2}

\begin{figure}[h]
\centering
    \begin{subfigure}{0.3\textwidth}
        \includegraphics[width=4cm, height=2cm]{Alphabet41393.jpg}
        \caption{Caption1}
        \label{fig:fig2}
    \end{subfigure}
    \begin{subfigure}{0.3\textwidth}
        \includegraphics[width=4cm, height=2cm]{Alphabet41393.jpg}
        \caption{Caption2}
        \label{fig:fig3}
    \end{subfigure}
    \caption{Caption for figure with 2 images}
\end{figure}
\FloatBarrier

\section{Dealing sections}
\todo{Move sections to their only file for include}
Praesent blandit blandit mauris. Praesent
lectus tellus, aliquet aliquam, luctus a, egestas a, turpis. Mauris 
lacinia loremsit amet ipsum. Nunc quis urna dictum turpis accumsan 
semper.
\subsection{Dealing with text}
\paragraph{This is a paragraph test}
\subparagraph{This is a subparagraph test}

Praesent blandit blandit mauris. Praesent
lectus tellus, aliquet aliquam, luctus a, egestas a, turpis. Mauris 
lacinia loremsit amet ipsum. Nunc quis urna dictum turpis accumsan 
semper.In figure~\ref{fig:fig2} on page~\pageref{fig:fig2}

\section*{Dealing with unnumbered sections}
Praesent blandit blandit mauris. Praesent
lectus tellus, aliquet aliquam, luctus a, egestas a, turpis. Mauris 
lacinia loremsit amet ipsum. Nunc quis urna dictum turpis accumsan 
semper.

\section{Dealing with input} 
Testing if the sections is inputted on the page correctly.

\section{Dealing with include} 
Testing if the sections is included correctly.

\section{Dealing with lists}

\paragraph{unordered list}
\renewcommand{\labelitemi}{$\blacksquare$}
\begin{itemize}
  \item One entry in the list
  \item Another entry in the list
\end{itemize}

\paragraph{ordered list}
\begin{enumerate}
  \item The labels consists of sequential numbers.
  \item The numbers starts at 1 with every call to the enumerate environment.
\end{enumerate}


\paragraph{nested list}
\renewcommand{\labelitemi}{\textbullet}
\begin{enumerate}
   \item The labels consists of sequential numbers.
   \begin{itemize}
     \item The individual entries are indicated with a black dot, a so-called bullet.
     \item The text in the entries may be of any length.
   \end{itemize}
   \item The numbers starts at 1 with every call to the enumerate environment.
\end{enumerate}

\section{Dealing with tables}
\todo{make more advanced tables}
The table \ref{table:table1} is an example of referenced \LaTeX elements.
\begin{table}[h]
\centering
\begin{tabular}{ | l | c | c | c | r |}
\hline
\multicolumn{5}{|c|}{Header} \\
\hline
Col1 & Col2 & Col3 & Col4 & Col5 \\ [0.5ex]
\hline
 cell1 & cell2 & cell3 & cell4 & cell5\\ 
 cell6 & cell7 & cell8 & cell9 & cell10\\  
 cell11 & cell12 & cell13 & cell14 & cell5 \\
 cell16 & cell17 & cell18 & cell19 & cell20\\
 cell21 & cell22 & cell23 & cell24 & cell25\\
 \hline
\end{tabular}
\caption{Table to test captions and labels}
\label{table:table1}
\end{table}

\section{Dealing with code}

\definecolor{dkgreen}{rgb}{0,0.6,0}
\definecolor{gray}{rgb}{0.5,0.5,0.5}
\definecolor{mauve}{rgb}{0.58,0,0.82}
\lstset{frame=tb,
  language=Java,
  aboveskip=3mm,
  belowskip=3mm,
  showstringspaces=false,
  columns=flexible,
  basicstyle={\small\ttfamily},
  numbers=none,
  numberstyle=\tiny\color{gray},
  keywordstyle=\color{blue},
  commentstyle=\color{dkgreen},
  stringstyle=\color{mauve},
  breaklines=true,
  breakatwhitespace=true,
  tabsize=3
}
\begin{lstlisting}[language=Java]
class Simple{  
    public static void main(String args[]){  
     System.out.println("Hello Java");  
    }  
}  
\end{lstlisting}

\section{Dealing with math equations}

The well known Pythagorean theorem \(x^2 + y^2 = z^2\) was 
proved to be invalid for other exponents. 
Meaning the next equation has no integer solutions:
\[ x^n + y^n = z^n \]
As you see, the way the equations are displayed depends on the delimiter, in this case [ ] and ( ).

\[ \frac12 \]

\[ \frac{2}{x+2} \]

\[ \frac{1+\frac{1}{x}}{3x + 2} \]

\[ \sqrt{3} \]

\[ \sqrt{x+\frac{1}{2}} \]

\[ \sqrt[3]{3} \]

\[ \sum_{i=1}^{\infty}\frac{1}{i} \]

\[ \prod_{n=1}^5\frac{n}{n-1} \]

\url{https://artofproblemsolving.com/wiki/index.php/LaTeX:Commands}

\section{Dealing Bibliography}
Let's cite! The Einstein's journal paper \cite{einstein} and the shakespeare1974riverside's book \cite{shakespeare1974riverside} are physics related items. 

\printbibliography[heading=bibintoc,
title={Bibliography Reference}]

\end{document}
